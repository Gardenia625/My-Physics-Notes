\usepackage{geometry}
\geometry{
    top=2.25cm,
    bottom=2.25cm,
    inner=2cm, % left
    outer=2cm, % right
		headheight=5ex, % 页眉高度
		headsep=5ex, % 页眉和正文间距
}
\raggedbottom % 防止 vbox underfull 警告

\usepackage{fancyhdr} % 用来设置页眉页脚
\renewcommand{\sectionmark}[1]{\markright{\S\,\thesection\quad#1}}
% 正文格式
\fancypagestyle{fancy}{
  \fancyhf{} % 清除页眉页脚格式
  \fancyhead[LE,RO]{\thepage} % 页码显示
  \fancyhead[CE]{\leftmark} % 偶页页眉中心 
  \fancyhead[CO]{\rightmark} % 奇页页眉中心
}
% 目录第一页及 chapter 第一页格式
\fancypagestyle{plain}{\fancyhf{}}
% 目录与前言格式
\fancypagestyle{toc}{ 
  \fancyhf{}
  \fancyhead[LE,RO]{\thepage\vspace{1mm}}
}
\renewcommand{\headrulewidth}{0pt} % 去掉页眉横线

\pagestyle{plain}
\usepackage{float}
\usepackage{amssymb}
\usepackage{amsmath}
\usepackage{amsfonts}
\numberwithin{figure}{chapter}
\usepackage{graphicx}
\usepackage{enumerate} % 多种样式的 enumerate 编号
%\usepackage{enumitem} % 多种样式的 enumerate 编号
\usepackage{bm} %希腊字母加粗
\usepackage{xcolor} % \textcolor{red}{123}
\usepackage{amsthm}
% 让定理的标题变为粗体
\makeatletter
\def\th@plain{%
  \thm@notefont{}% same as heading font
  \itshape % body font
}
\def\th@definition{%
  \thm@notefont{}% same as heading font
  \normalfont % body font
}
\makeatother
\newtheorem{theorem}{定理}[chapter]
\newtheorem{axiom}[theorem]{公理}
\newtheorem{lemma}[theorem]{引理}
\newtheorem{definition}[theorem]{定义} 
\newtheorem{proposition}[theorem]{命题}
\newtheorem{corollary}[theorem]{推论}
\newtheorem{example}[theorem]{例}
\newtheorem*{remark}{注}

\newtheorem{quantumaxiom}{量子力学公理}

\usepackage{hyperref} % 文档内跳转
\usepackage{xpatch}
% 去掉定理编号后面的点
%\makeatletter
%\AtBeginDocument{\xpatchcmd{\@thm}{\thm@headpunct{.}}{\thm@headpunct{}}{}{}}
%\makeatother

% 证明粗体
\xpatchcmd{\proof}{\itshape}{\normalfont\proofnamefont}{}{}
\newcommand{\proofnamefont}{\bfseries}
% 提升 chi 的高度
\usepackage{mathtools}
\DeclareRobustCommand{\rchi}{{\mathpalette\irchi\relax}}
\newcommand{\irchi}[2]{\raisebox{\depth}{$#1\chi$}}
% 优化 forall 和 exists 后面的缩进
\let\oldforall\forall
\let\forall\undefined
\DeclareMathOperator{\forall}{\oldforall}
\let\oldexists\exists
\let\exists\undefined
\DeclareMathOperator{\exists}{\oldexists}

\usepackage{indentfirst}
\setlength{\parindent}{2em} %首行缩进
\usepackage{setspace} % 调整行间距
% 用来调整行间距, 不要与 setspace 混用
% \renewcommand{\baselinestretch}{1.25} 

% \keepline 可以用来让以行间公式开头的定理或证明不另起一行
% 使用方法:
% \begin{theorem}[title]\keepline
% ...
% \end{theorem}
\newcommand{\keepline}{\leavevmode\setlength{\abovedisplayskip}{0pt}\vspace{-\baselineskip}}