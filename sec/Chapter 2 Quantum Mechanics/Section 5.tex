\section{角动量}
\subsection{轨道角动量}
取量子希尔伯特空间为 $ L^2(\mathbb{R}^3) $. 考虑经典相空间上的角动量
\[ x\times p=\left( \begin{matrix}
    x_2p_3-x_3p_2\\ 
    x_3p_1-x_1p_3\\ 
    x_1p_2-x_2p_1
\end{matrix} \right), \]
其三个分量对应的算子为
\begin{align*}
    L_1 :&= X_2P_3-X_3P_2,\\ 
    L_2 :&= X_3P_1-X_1P_3,\\ 
    L_3 :&= X_1P_2-X_2P_1.
\end{align*}
这些算子叫做{\bf 轨道角动量算子} (orbital angular momentum operator).
\vspace{1ex}
\begin{proposition}\keepline
    \[ \begin{aligned}
        [L_1,L_2] &= \mathrm{i}\hbar L_3,\\ 
        [L_2,L_3] &= \mathrm{i}\hbar L_1,\\ 
        [L_3,L_1] &= \mathrm{i}\hbar L_2.
    \end{aligned} \]
\end{proposition}
定义角动量模长平方所对应的算子为
\[ L^2:=L_1^2+L_2^2+L_3^2.\vspace{1ex} \]
\begin{proposition}\keepline
    \[ [L^2,L_1]=[L^2,L_2]=[L^2,L_3]=0. \]
\end{proposition}
\vspace{1ex}
与处理简谐振子的方式类似, 我们定义两个不自伴的辅助算子:
\begin{enumerate}
    \item[$ \bullet $] {\bf 升阶算子} $ L_+:=L_1+\mathrm{i}L_2 $,
    \item[$ \bullet $] {\bf 降阶算子} $ L_-:=L_1-\mathrm{i}L_2 $. 
\end{enumerate}
\vspace{0ex}
\begin{proposition}\keepline
    \[ \begin{aligned}
        [L_3,L_{\pm}] &= \pm\hbar L_{\pm},\\ 
        [L^2,L_{\pm}] &= 0,\\ 
        [L_+,L_-] &= 2\hbar L_3.
    \end{aligned} \]
\end{proposition}
\vspace{0ex}
设 $ L_3f=\mu f $, 则
\[ L_3(L_{\pm}f)=L_{\pm}L_3 f\pm\hbar L_{\pm}f=(\mu\pm\hbar)L_{\pm}f. \]
这说明 $ L_+ $ 可将 $ L_3 $ 的特征向量提升为特征值加 $ \hbar $ 的特征向量, 而 $ L_- $ 则对应地将其降低为特征值减 $ \hbar $ 的特征向量. 由于 $[L^2,L_3]=0$, $f$ 也是 $L^2$ 的特征向量. 设 $ L^2 f=\lambda f $, 则由 $[L^2,L_{\pm}]=0$ 可知
\[ L^2(L_{\pm}f)=\lambda L_{\pm}f, \]
这说明
\[ \langle L^2 \rangle_{\pm f}=\langle L_{\pm}f,L^2(L_{\pm} f) \rangle=\lambda. \]
设 $\tilde{f}={L_{\pm}^k}f$, 则由
\begin{align*}
    \lambda&=\langle L^2\rangle_{\tilde{f}}=\langle L_1^2\rangle_{\tilde{f}}+\langle L_2^2\rangle_{\tilde{f}}+\langle L_3^2\rangle_{\tilde{f}}\\ 
    &\geqslant\langle L_3^2\rangle_{\tilde{f}}\\ 
    &=\langle L_3 \tilde{f},L_3 \tilde{f}\geqslant 0
\end{align*}
知特征值不能被无限地提升或降低. 设特征值最高可被提升为 $ \sigma:=\mu+k\hbar $ 并记
\begin{align*}
    f_0:&=(L^+)^kf,\\ 
    f_j:&=(L^-)^jf_0,\quad j\geqslant 1.
\end{align*}
则 $ L_3f_j=(\sigma-j\hbar)f_j $. 利用 $ [L_+,L_-]=2\hbar L_3 $ 可归纳地证明
\[ L_+f_j=j\hbar(2\sigma+\hbar-j\hbar)f_{j-1},\quad j\geqslant 1. \]
设最大的使得 $ f_j\neq 0 $ 的指标为 $ N $, 则
\[ 0=L^+f_{N+1}=(N+1)\hbar(2\sigma-N\hbar)f_N, \]
这说明
\[ \sigma=\frac{1}{2}N\hbar.\vspace{2ex} \]
\begin{proposition}\keepline
    \[ L^2=L_{\pm}L_{\mp}+L_3^2\mp\hbar L_3. \]
\end{proposition}
\vspace{0ex}
\begin{proof}\keepline
    \begin{align*}
        L_{\pm}L_{\mp} &= (L_1\pm\mathrm{i}L_2)(L_1\mp\mathrm{i}L_2)\\ 
        &= L_1^2+L_2^2\mp[L_1,L_2]\\ 
        &= L^2-L_3^2\pm\hbar L_3.\qedhere
    \end{align*}
\end{proof}
记 $ \displaystyle l=\frac{N}{2} $, 则
\[ L^2f_0=(L_-L_++L_3^2+\hbar L_3)f_0=\hbar^2 l(l+1)f_0, \]
因此
\[ \lambda = \hbar^2 l(l+1). \]

最后我们不加证明地指出 $ N $ 可以取遍所有偶数 (但不能是奇数), 且 $ L^2 $ 与 $ L_3 $ 的特征向量相同, 形如 $ Y_l^m(\theta,\varphi)f(|r|) $, 其中 $ Y^m_l(\theta,\varphi) $ 为{\bf 球谐函数}:
\[ L^2Y^m_l=\hbar^2l(l+1)Y^m_l,\quad L_3Y^m_l=\hbar mY^m_l, \]
\[ l=0,1,2\dots,\quad m=-l,-l+1,\dots,l-1,l. \]
其中 $ l $ 叫做{\bf 角量子数} (azimuthal quantum number), $ m $ 叫做{\bf 磁量子数} (magnetic quantum number).

{\bf 球谐函数} (spherical harmonics) 既可以定义为 $ S^2 $ 上{\bf 拉普拉斯方程}
\[ \nabla^2f=\frac{1}{r^2}\frac{\partial}{\partial r}\left( r^2\frac{\partial f}{\partial r} \right)+\frac{1}{r^2\sin\theta}\frac{\partial}{\partial\theta}\left( \sin\theta\frac{\partial f}{\partial\theta} \right)+\frac{1}{r^2\sin^2\theta}\frac{\partial^2 f}{\partial\varphi^2}=0 \]
的解, 也可以定义为将{\bf 调和多项式} (harmonic polynomial) 限制在单位球面上所得到的函数. 调和多项式是满足拉普拉斯方程 $ \Delta p=0 $ 的多项式 $ p:\mathbb{R}^3\to\mathbb{C} $, 其中 $ \Delta:=\nabla^2 $ 为{\bf 拉普拉斯算子}.
\subsection{角动量作为 \texorpdfstring{$ \mathfrak{so}(3) $}{so3} 的表示}
在量子力学中, 粒子除了有轨道角动量, 还有自旋角动量. 自旋角动量会产生磁场, 在经典力学中没有与之对应的物理量. 为了引入自旋角动量, 我们需要先从另一个角度来看轨道角动量. 在此, 我们直接指出: $ \mathrm{SO}(3) $ 的表示对应了轨道角动量, $ \mathrm{SU}(2) $ 的表示对应了自旋角动量.

有关李群李代数的基本概念可见附录 \ref{lie}. 特殊正交群 $ \mathrm{SO}(3) $ 是所有行列式为 $ 1 $ 的 $ 3\times 3 $ 正交矩阵构成的群 ($ \mathrm{SO}(3) $ 是连通的), 其对应的李代数为
\[ \mathfrak{so}(3)=\left\{ \left( \begin{matrix}
    0 & a & b\\ 
    -a & 0 & c\\ 
    -b & -c & 0
\end{matrix} \right) \;\middle|\; a,b,c\in\mathbb{R} \right\}. \]
取 $ \mathfrak{so}(3) $ 的一组基 
\[ F_1:=\left( \begin{matrix}
    0 & 0 & 0\\ 
    0 & 0 & -1\\ 
    0 & 1 & 0
\end{matrix} \right),\quad F_2:=\left(\begin{matrix}
    0 & 0 & 1\\ 
    0 & 0 & 0\\ 
    -1 & 0 & 0
\end{matrix}\right),\quad F_3:=\left(\begin{matrix}
    0 & -1 & 0\\ 
    1 & 0 & 0\\ 
    0 & 0 & 0
\end{matrix}\right). \]
容易验证
\begin{align*}
    [F_1,F_2] &= F_3,\\ 
    [F_2,F_3] &= F_1,\\ 
    [F_3,F_1] &= F_2.
\end{align*}

\begin{proposition}
    \label{SO3L2}
    对于 $ R\in\mathrm{SO}(3) $, 定义算子 $ \Pi(R):L^2(\mathbb{R}^3)\to L^2(\mathbb{R}^3) $ 为
    \[ (\Pi(R)\psi)(x):=\psi(R^{-1}x). \]
    这样定义的 $ \Pi:\mathrm{SO}(3)\to\mathrm{U}(L^2(\mathbb{R}^3)) $ 是一个酉表示, 见定义 \ref{unitary representation 1}.

    对于命题 \ref{unitary representation 2} 中定义的 $ \pi $, $ \pi(F_j) $ 的定义域包含 $ \mathcal{S}(\mathbb{R}^3) $, 且在 $ \mathcal{S}(\mathbb{R}^3) $ 上有
    \[ L_j=\mathrm{i}\hbar\pi(F_j), \]
    其中 $ L_j $ 是轨道角动量.
\end{proposition}

因此求 $ L_j $ 的特征值和特征向量就是求 $ \pi(F_j) $ 的特征值和特征向量. 为了解决这个问题, 我们可以先考虑 $ \mathfrak{so}(3) $ 的有限维表示 $ \pi $, 求出此时 $ \pi(F_j) $ 的特征值和特征向量. 仿照前文中对轨道角动量所做的分析, 可以证明下述定理.

\begin{theorem}
    \label{so3}
    令 $ \pi:\mathfrak{so}(3)\to\mathfrak{gl}(V) $ 为有限维不可约表示, 并在 $ V $ 上定义三个算子:
    \begin{align*}
        L_+ &= \mathrm{i}\pi(F_1)-\pi(F_2),\\ 
        L_- &= \mathrm{i}\pi(F_1)+\pi(F_2),\\ 
        L_3 &= \mathrm{i}\pi(F_3).
    \end{align*}
    令 $ l=\frac{1}{2}(\mathrm{dim}V-1) $, 则 $ \mathrm{dim}V=2l+1 $ 且存在 $ V $ 的一组基 $ v_0,v_1,\dots,v_{2l} $ 使得
    \begin{align*}
        L_3v_j &= (l-j)v_j,\\ 
        L_-v_j &= \begin{cases}
            v_{j+1}, & j<2l,\\ 
            0, & j=2l,
        \end{cases}\\ 
        L_+v_j &= \begin{cases}
            j(2l+1-j)v_{j-1}, & j>0,\\ 
            0, & j=0.
        \end{cases}
    \end{align*}
\end{theorem}

下述定理指出当 $ l $ 取整数和半整数时都有对应的 $ \mathfrak{so}(3) $ 不可约表示

\begin{theorem}
    对任意 $ l=0,\frac{1}{2},1,\frac{3}{2},\dots $, 存在 $ \mathfrak{so}(3) $ 的 $ 2l+1 $ 维不可约表示, 且任意两个 $ 2l+1 $ 维不可约表示是同构的.
\end{theorem}

下述定理指出只有 $ l $ 为整数是才有与之对应的 $ \mathrm{SO}(3) $ 的表示.

\begin{proposition}
    若 $ l $ 为整数, 则存在表示 $ \Pi_l:\mathrm{SO}(3)\to\mathrm{GL}(V) $ 使得
    \[ \Pi_l\left(e^{tX}\right)=e^{t\pi_l(X)}. \]
    若 $ l $ 为半整数, 则不存在这样的 $ \mathrm{SO}(3) $ 的表示.
\end{proposition}

命题 \ref{SO3L2} 定义了 $ \mathrm{SO}(3) $ 在 $ L^2(\mathbb{R}^3) $ 上的作用 $ \Pi(R)\psi(x)=\psi(R^{-1}x) $. 类似地, 可以定义 $ \mathrm{SO}(3) $ 在 $ L^2(S^2) $ 上的作用 $ \Pi(R)\psi(x)=\psi(R^{-1}x),\ x\in S^2 $.

\begin{definition}
    对于非负整数 $ l $, $ \mathbb{R}^3 $ 上所有 $ l $ 次的调和多项式限制到 $ S^2 $ 上后可以得到 $ L^2(S^2) $ 的一个线性子空间 $ V_l $, 称其为 $ l $ 次球谐函数构成的空间.
\end{definition}

\begin{proposition}
    $ \mathrm{dim}V_l=2l+1 $, $ V_l $ 在 $ \mathrm{SO}(3) $ 作用下不变且在该作用下不可约 (不存在非平凡不变子空间). 所有 $ V_l $ 相互正交且
    \[ L^2(S^2)=\bigoplus_{l=0}^\infty V_l. \]
\end{proposition}

将命题 \ref{SO3L2} 中所有 $ \Pi(R) $ 限制在 $ V_l $ 上会得到不可约表示 $ \mathrm{SO}(3)\to \mathrm{GL}(V_l) $, 由定理 \ref{GGgg} 可诱导出不可约表示 $ \mathfrak{so}(3)\to\mathfrak{gl}(V_l) $, 而 $ F_j $ 在该表示下的像正是命题 \ref{SO3L2} 中 $ \pi(F_j) $ 在 $ V_l $ 上的限制, 因此再使用定理 \ref{so3} 就得到了我们想要的特征值, 通过分析的方法还可以求得特征向量. 因为所有 $ V_l $ 的直和为 $ L^2(S^2) $, 我们可以通过这个方法求出 $ L_j $ 在 $ L^2(S^2) $ 上的所有特征值和特征向量. 最后, 下述结论可将结论推广到 $ L^2(\mathbb{R}^3) $ 上.

\begin{proposition}
    设 $ f:(0,\infty)\to\mathbb{C} $ 是可测函数且满足
    \[ \int_{0}^{\infty}|f(r)|^2r^{2l+2}\,\mathrm{d}r<\infty. \]
    用 $ V_{l,f}\subset L^2(\mathbb{R}^3) $ 表示所有形如
    \[ \psi(x)=p(x)f(|x|) \]
    的 $ \psi(x) $ 构成的空间, 其中 $ p\in V_l $. 则 $ V_{l,f} $ 在 $ \mathrm{SO}(3) $ 作用下不变且不可约. 反之, 每个 $ L^2(\mathbb{R}^3) $ 的有限维不可约的 $ \mathrm{SO}(3) $ 不变子空间都是某个 $ V_{l,f} $.     
\end{proposition}

\subsection{自旋角动量}
特殊酉群 $ \mathrm{SU}(2) $ 是所有行列式为 $ 1 $ 的 $ 2\times 2 $ 酉矩阵构成的群, 它对应的李代数为 
\[ \mathfrak{su}(2)=\left\{ \left(\begin{matrix}
    \mathrm{i} a & b+\mathrm{i}c\\ 
    -b+\mathrm{i}c & -\mathrm{i}a
\end{matrix}\right) \;\middle|\; a,b,c\in\mathbb{R} \right\}. \]
取 $ \mathfrak{su}(2) $ 的一组基

\[ E_1=\frac{1}{2}\left(\begin{matrix}
    \mathrm{i} & 0\\ 
    0 & -\mathrm{i}
\end{matrix}\right),\quad E_2=\frac{1}{2}\left(\begin{matrix}
    0 & 1\\ 
    -1 & 0
\end{matrix}\right),\quad E_3=\frac{1}{2}\left(\begin{matrix}
    0 & \mathrm{i}\\ 
    \mathrm{i} & 0
\end{matrix}\right), \]
不难看出 $ \mathfrak{su}(2) $ 与 $ \mathfrak{so}(3) $ 同构. 
\begin{remark}
    但 $ \mathrm{SU}(2) $ 与 $ \mathrm{SO}(3) $ 不同构, $ \mathrm{SU}(2)/\{I,-I\}\cong\mathrm{SO}(3) $.
\end{remark}

\begin{definition}[自旋]
    每种粒子都有对应的{\bf 自旋} (spin), 自旋可以取非负整数和半整数. 对于自旋为 $ s $ 的粒子, 它对应的量子希尔伯特空间为 $ L^2(\mathbb{R}^3)\otimes V_s $, 其中 $ \otimes $ 表示张量积, $ V_s $ 是 $ 2s+1 $ 维不可约表示 $ \pi:\mathrm{SU}(2)\to\mathrm{GL}(V_s) $ 中的 $ V_s $.
\end{definition}

\begin{definition}
    拥有整数自旋的粒子叫做{\bf 玻色子} (boson), 拥有半整数自旋的粒子叫做{\bf 费米子} (fermion).
\end{definition}

由于 $ \mathrm{SU}(2) $ 是单连通的, 由定理 \ref{GGgg} 和定理 \ref{ggGG} 可知 $ \mathrm{SU}(2) $ 的表示与 $ \mathfrak{su}(2) $ 的表示存在一一对应. 因此研究 $ \mathrm{SU}(2) $ 的表示就是在研究 $ \mathfrak{su}(2) $ 的表示.

若粒子对应的表示为 $ \pi:\mathfrak{su}(2)\to\mathfrak{gl}(V_l) $, 定义{\bf 自旋角动量}为
\begin{align*}
    S_j = \mathrm{i}\hbar\pi(E_j),\quad j=1,2,3.
\end{align*}
它们与轨道角动量有相同的对易关系
\begin{align*}
    [S_1,S_2] &= \mathrm{i}\hbar S_3,\\ 
    [S_2,S_3] &= \mathrm{i}\hbar S_1,\\ 
    [S_3,S_1] &= \mathrm{i}\hbar S_2.
\end{align*}
仿照先前的分析, 定义
\begin{align*}
    S^2&=S_1^2+S_2^2+S_3^2,\\ 
    S_{\pm}&=S_1\pm\mathrm{i}S_2,
\end{align*}
并通过相同的分析可以可以解得
\begin{align*}
    S^2|s,m\rangle&=\hbar^2s(s+1)|s,m\rangle,\\
    S_3|s,m\rangle&=\hbar m|s,m\rangle,\\ 
    S_{\pm}|s,m\rangle&=\hbar\sqrt{s(s+1)-m(m\pm 1)}|s,m\pm 1\rangle,
\end{align*}
其中 $ s=0,\frac{1}{2},1,\frac{3}{2},\dots $ 为自旋, $ m=-s,-s+1,\dots,s-1,s $ 为{\bf 自旋量子数} (spin quantum number). 
\begin{remark}
    这里我们使用了物理学家的记号, 将参数 $ s,m $ 对应的特征向量记为 $ |s,m\rangle $.
\end{remark}

最后, 对于自旋为 $ s $ 的粒子, 可将其{\bf 总角动量}定义为 $ L^2(\mathbb{R}^3)\otimes V_s $ 上的算子
\[ J_j=L_j\otimes I+I\otimes S_j,\quad j=1,2,3. \]

\begin{example}[$\frac{1}{2}$ 自旋]
    当 $ s=\frac{1}{2} $ 时, 有 $ S_j=\frac{\hbar}{2}\sigma_j $, $ j=1,2,3 $, 其中
    \[ \sigma_1=\left(\begin{matrix}
        0 & 1\\ 
        1 & 0
    \end{matrix}\right),\quad\sigma_2=\left(\begin{matrix}
        0 & -\mathrm{i}\\ 
        \mathrm{i} & 0
    \end{matrix}\right),\quad\sigma_3=\left(\begin{matrix}
        1 & 0\\ 
        0 & -1
    \end{matrix}\right) \]
    为{\bf 泡利矩阵}. 通过简单地计算可得
    \[ S^2=\frac{3}{4}\hbar^2 \left(\begin{matrix}
        1 & 0\\ 
        0 & 1
    \end{matrix}\right),\quad S_+=\hbar \left(\begin{matrix}
        0 & 1\\ 
        0 & 0
    \end{matrix}\right),\quad S_-=\hbar \left(\begin{matrix}
        0 & 0\\ 
        1 & 0
    \end{matrix}\right). \]
    此时 $ S^2 $ 和 $ S_3 $ 有两个特征向量 
    \[ \left|\frac{1}{2},\frac{1}{2}\right\rangle=\left(\begin{matrix}
        1 \\ 0
    \end{matrix}\right),\quad  \left|\frac{1}{2},-\frac{1}{2}\right\rangle=\left(\begin{matrix}
        0 \\ 1
    \end{matrix}\right),\] 
    物理学家一般将它们分别记作 $ \left|\uparrow\right\rangle $ 和 $ \left|\downarrow\right\rangle $.
\end{example}